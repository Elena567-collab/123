% !TEX program = xelatex
\documentclass[12pt,a4paper,twoside]{article}
\usepackage{multicol}
\usepackage{geometry}
\usepackage{fontspec}       
\usepackage{fancyhdr}
\usepackage{setspace}
\usepackage{abstract}
\usepackage{indentfirst}
\usepackage{titlesec}
\usepackage{graphicx}
\usepackage{caption}
\usepackage{amsmath}
\usepackage{array}
\usepackage{multirow}
\usepackage{makecell}
\usepackage{booktabs}
\usepackage{microtype} 

% ==================== НАСТРОЙКИ ЯЗЫКА И ШРИФТОВ ====================

% Основные шрифты
\setmainfont{Times New Roman}
\setsansfont{Arial}
\setmonofont{Courier New}

% Установка математического шрифта (должен быть ПЕРВЫМ среди math font)
\usepackage{unicode-math}
\setmathfont{Latin Modern Math} % или другой OpenType Math шрифт

\usepackage[russian]{babel}

% Настройки для улучшения переносов
\pretolerance=1000    % Предварительная терпимость к плохим строкам
\tolerance=2000       % Максимальная терпимость к плохим строкам
\emergencystretch=10pt % Дополнительное растяжение в экстренных случаях
\hbadness=10000       % Игнорировать предупреждения о плохих строках
\hyphenpenalty=50     % Штраф за перенос (меньше = больше переносов)
\exhyphenpenalty=50   % Штраф за перенос после дефиса

% Разрешить переносы после дефисов в составных словах
\babelhyphenation[russian]{}

% ==================== НАСТРОЙКА СТРАНИЦЫ ====================
\geometry{
	a4paper,
	left=20mm,
	right=15mm,
	top=25mm,
	bottom=25mm,
	headheight=14.5pt,
	headsep=10pt,
	footskip=15pt, % для места в нижнем колонтитуле
	columnsep=7mm
}

% ==================== НАСТРОЙКА ЗАГОЛОВКОВ ====================
\titleformat{\section}{\normalfont\Large}{\thesection}{1em}{}
\titleformat{\subsection}{\normalfont\large\bfseries}{\thesubsection}{1em}{}
\titlespacing*{\section}{0pt}{10pt}{5pt}
\titlespacing*{\subsection}{0pt}{8pt}{3pt}

% ==================== НАСТРОЙКА КОЛОНТИТУЛОВ ====================
\pagestyle{fancy}
\fancyhf{}
\renewcommand{\headrulewidth}{0pt}
\renewcommand{\footrulewidth}{0pt}

% Для ЧЕТНЫХ страниц (левых) - номер слева вверху, надпись слева внизу
\fancyhead[LE]{\thepage} % номер слева вверху
\fancyfoot[RE]{\footnotesize\textit{ФИЗИКА ЗЕМЛИ № 6 2024}} % надпись слева внизу

% Для НЕЧЕТНЫХ страниц (правых) - номер справа вверху, надпись справа внизу
\fancyhead[RO]{\thepage} % номер справа вверху
\fancyfoot[LO]{\footnotesize\textit{ФИЗИКА ЗЕМЛИ № 6 2024}} % надпись справа внизу

% Стиль для первой страницы (без номеров и надписей в колонтитулах)
\fancypagestyle{firstpage}{
	\fancyhf{} % очистить все колонтитулы
	\renewcommand{\headrulewidth}{0pt}
	\renewcommand{\footrulewidth}{0pt}
}

% ==================== ПРОЧИЕ НАСТРОЙКИ ====================
\onehalfspacing
\setlength{\parindent}{0.25cm}

% Команда для двойной линии
\newcommand{\doubleline}{%
	\par\noindent
	\rule{\linewidth}{0.4pt}\par\vspace{-4ex}
	\rule{\linewidth}{0.4pt}\par
}

\begin{document}
	
	% ==================== ПЕРВАЯ СТРАНИЦА ====================
	\thispagestyle{firstpage}
	\onecolumn
	
	\begin{flushleft}
		\noindent
		\fontsize{10}{18}\selectfont
		\textit{ФИЗИКА ЗЕМЛИ, 2024, № 6, с. 1–28}
		\doubleline
		\noindent
		\fontsize{10}{18}\selectfont
		\textit{УДК 551.521}
	\end{flushleft}
	
	\vspace{0,5cm}
	
	\begin{center}
		% Название статьи
		\fontsize{18}{22}\selectfont
		\textbf{ИССЛЕДОВАНИЕ ТЕРМОДИНАМИКИ ПЛАЗМЫ СОЛНЕЧНОГО ВЕТРА В КОРОНЕ СОЛНЦА ПО ЗАРЯДОВОМУ СОСТОЯНИЮ ТЯЖЕЛЫХ ИОНОВ}
		
		\vspace{1cm}
		
		% Авторы
		\fontsize{11}{18}\selectfont
		\textbf{
			© 2024 г. \quad Ф. Ф. Горяев$^{1*}$, В. А. Слемзин$^{1**}$}
		
		\vspace{0.5cm}
		
		\fontsize{10}{16}\selectfont\textit{
			$^1$Физический институт им. П.Н. Лебедева РАН, Москва, Россия}
		
				% Контактная информация
		\noindent
		\fontsize{10}{16}\selectfont
		$^*$Электронный адрес: goryaev\_farid@mail.ru \\
		$^{**}$Электронный адрес: slemzinva@lebedev.ru \\
		
		% Даты
		\fontsize{10}{16}\selectfont
		Поступила в редакцию 03.04.2024 г. \\
		После доработки 06.06.2024 г.\\
		Принята к публикации 06.06.2024 г.
	\end{center}
	
	\vspace{0.5cm}
	
	
		\noindent Термодинамика плазмы солнечного ветра (СВ) в короне Солнца определяется энергетическим обменом с внешними источниками и может быть изучена, если известна информация о физических параметрах плазмы, таких как температура, плотность, скорости потоков СВ и др. Ранее Паркер показал, что в рамках одножидкостной модели состояние плазмы СВ может быть описано с помощью политропной функции, в которой давление $p$ и плотность $\rho$ связаны соотношением $p/\rho^\gamma = \text{const}$ с индексом политропы $\gamma$. В современных МГД-моделях применение политропной функции вместо приближенного описания механизмов нагрева плазмы значительно ускоряет расчет. Значения индекса $\gamma$ могут быть получены с помощью параметров плазмы СВ, но для потоков СВ, движущихся в направлении Земли, измерение таких параметров представляет определенные трудности. В настоящей работе рассматривается метод определения индекса политропы $\gamma$ для потоков СВ на стадии расширения в короне по измеряемым ``на месте'' ионным параметрам плазмы СВ: среднему заряду ионов железа $\langle Q_{\text{Fe}}\rangle$ и отношению плотностей ионов кислорода O$^{7+}$/O$^{6+}$. Связь между ионными параметрами и индексом $\gamma$ устанавливается на основе решения уравнений баланса для процессов ионизации и рекомбинации ионов в плазме СВ. По гистограммам ионных параметров СВ, измеренных прибором ACE/SWICS в 2010 г., получены средние значения $\gamma$ в короне на высотах $\approx 1$--7 солнечных радиусов для потоков медленного СВ, быстрого СВ из корональных дыр и межпланетных корональных выбросов массы.
		
		\medskip
		\noindent
		\textbf{Ключевые слова:} солнечный ветер, солнечная корона, ионный состав солнечного ветра.
	
	
		
	% ==================== ВТОРАЯ СТРАНИЦА И ДАЛЕЕ ====================
	\newpage
	\twocolumn
	
	\section*{1.	ВВЕДЕНИЕ}
	\setlength{\parindent}{1.25cm} % Отступ первой строки (1.25 см)
	\setlength{\parskip}{0pt}
	В процессе формирования потоков солнечного ветра (СВ) в короне происходит динамическое изменение параметров плазмы СВ под влиянием обмена энергией с внешними источниками. Информация об этих процессах важна как в теоретическом плане, так и для практических применений, прежде всего, для моделирования параметров СВ и прогнозирования его воздействия на околоземную космическую среду.	
	Трудность моделирования СВ в короне Солнца связана, в первую очередь, с особенностями структур, из которых формируется поток СВ, и сложностью оценки локальных параметров плазмы внутри короны, которые, в основном, определяются по спектральному излучению этих структур. Для источников на солнечном диске данные оптических измерений соответствуют интегральным интенсивностям вдоль линии наблюдения, поэтому локальные значения параметров плазмы в короне могут быть определены только косвенно через решение обратной задачи деконволюции, которая относится к классу некорректных задач, или путем сопоставления интегральных модельных параметров с экспериментально измеренными (см., например, \cite{16}). 
	
	Отдельные прямые измерения локальных параметров плазмы внутри короны на минимальном расстоянии менее 20 солнечных радиусов стали возможны в последнее время благодаря наблюдениям с помощью Солнечного зонда «Паркер» (Parker Solar Probe: \cite{18, 19}).
	
	Среди подходов, разработанных для моделирования энергетического баланса СВ, часто используется рассмотренная еще Паркером (\cite{20, 21}) политропная модель СВ – гидродинамическая одномерная модель радиального истечения плазмы из короны с политропным законом расширения, которая описывает переход от дозвукового течения плазмы в короне к сверхзвуковому в гелиосфере. В рамках этой модели соотношение между давлением $p$ и массовой плотностью $\rho$ одноатомного идеального газа описывается соотношением 
	
	\begin{equation}
		\frac{p}{\rho^\gamma} = \text{const},
		\label{eq:poly}
	\end{equation}
	
	где $\gamma$ – показатель (индекс) политропы.
	
	В первоначальной модели Паркера считалось, что плазма в короне изотермическая, что соответствует значению $\gamma = 1$. В гелиосфере, при отсутствии источников нагрева, показатель политропы расширяющейся и охлаждающейся плазмы СВ полагался равным адиабатическому значению $\gamma = 5/3$. Такая модель качественно описывала формирование стационарного потока СВ, однако параметры плазменного потока около Земли заметно отличались от измеряемых. 
	
	Согласование модельных параметров у Земли с измерениями температуры и плотности околосолнечной плазмы, опирающимися на оптические наблюдения солнечных затмений, привело к значению $\gamma$ в короне, близкому к значению 1.1 (\cite{21}). Соответствующий поток энергии в короне, по Паркеру, обусловлен теплопроводностью электронной компоненты.
	
	В дальнейшем, гидродинамические модели стационарного СВ развивались за счет включения энергетических соотношений в различных модификациях, содержащих разные типы источников энергии и тепла. Однако эти модели, основанные на использовании теоретически предполагаемых значений параметров плазмы в короне, как и учет магнитного давления (\cite{26}), не устранили расхождений модельных расчетов температуры, плотности и скорости в СВ с реально измеряемыми вблизи Земли (\cite{29}). 
	
	Частично согласование между плотностью и температурой потоков плазмы в рамках одножидкостной сферически симметричной модели улучшалось за счет самосогласованного рассмотрения условий в нижней короне и СВ в гелиосфере с волновыми источниками энергии–импульса и радиальным магнитным полем (\cite{10, 13}).
	
	Эпизодические измерения параметров СВ в гелиосфере с помощью космических аппаратов показали, что на разных масштабах расстояний индексы политропы могут различаться. В работе \cite{26} рассматривались значения индекса $\gamma$, получаемые из соотношения между температурой и плотностью СВ, измеренных на космическом аппарате Helios на расстояниях от Солнца 0.3–1 а.е. (а.е. – астрономическая единица). Среднее значение индекса $\gamma$ составило $1.46 \pm 0.04$ без учета магнитного давления и $1.58 \pm 0.06$ с учетом магнитного давления. При этом диапазон значений $\gamma$ слабо зависел от скорости потока СВ.
	
	Измерения на зонде Паркер (\cite{19}) показали, что в среднем на интервале расстояний 0.2–0.8 а.е. значение $\gamma$ близко к адиабатическому. В то же время, на промежуточных участках размером порядка 0.1 а.е. были отмечены волнообразные колебания с максимальным значением $\gamma \approx 2.7$, что может быть связано с изменением степеней свободы или какими-то дополнительными механизмами диссипации энергии протонов.
	
	В работе \cite{15} приведены данные измерений на космическом аппарате Wind на расстоянии 1 а.е., по которым в период двух последних солнечных циклов (1995–2017 гг.) было получено значение $\gamma$, близкое к 1.86. Измерения радиальных зависимостей температуры и плотности для быстрого СВ на зонде Паркер (0.1 а.е.) и обсерватории Solar Orbiter (0.9 а.е.) в период между 15 июля и 15 октября 2020 г. привели к значению $\gamma = 1.517$. Отличие индекса от адиабатического авторы связывают с дополнительным нагревом за счет альвеновских флуктуаций (\cite{22}).
	
	Для построения моделей формирования СВ в солнечной короне необходимо знать значения параметров плазмы на расстояниях нескольких солнечных радиусов от поверхности Солнца. Поскольку возможности прямых измерений в короне затруднены, в большинстве случаев моделирование СВ выполняется по упрощенной схеме: сначала определяются параметры плазмы в точке максимума температуры в короне, которые далее используются как исходные для продолжения расчета в гелиосферной части, где расширение потока СВ рассматривается как адиабатическое. 
	
	Такая схема расчета позволяет получить параметры СВ по порядку величины на расстоянии 1 а.е., близкие к среднестатистически измеряемым по всему потоку СВ, но не учитывает различий для разных типов СВ. Более точный расчет может быть выполнен, если для данного типа потока определены радиальные профили параметров СВ в короне или их связь, описываемая с помощью индекса политропы.	
	Упомянутые выше гидродинамические модели лучше всего пригодны для описания квазистационарных потоков СВ: фонового медленного СВ (МСВ) и высокоскоростного СВ (ВСВ) из корональных дыр. Для описания межпланетных корональных выбросов массы (МКВМ) разрабатываются более сложные численные МГД-модели (см., например, \cite{1}). 
	
	Такие МГД-модели как WSA-Enlil+Cone (\cite{17}) и EUPHORIA (\cite{23}) рассматривают по отдельности две стадии образования СВ: корональную и гелиосферную. В корональной части рассматривается формирование и ускорение плазмы СВ в условиях расширяющегося потока магнитного поля, рассчитываемого в потенциальном или бессиловом приближении (см., например, \cite{24}). 
	
	При этом в расчет включаются различные механизмы нагрева плазмы: электронная теплопроводность, обмен энергией между ионами и электронами, нагрев за счет диссипации альвеновских волн и турбулентности и др. (см. обзор \cite{12}). Учет в МГД-моделях СВ потоков массы и энергии с помощью соответствующих уравнений, включающих указанные механизмы, значительно усложняет численное моделирование. 
	
	Однако без потери точности приближенные описания механизмов нагрева и охлаждения могут быть заменены политропной функцией, связывающей давление и плотность плазмы, существенно упрощающей и ускоряющей расчет (\cite{30}), что важно для оперативного прогнозирования параметров СВ на выходе из короны.
	
	Для использования политропной модели необходимо знание ее индекса $\gamma$ для разных типов СВ в конкретных условиях. Работы по экспериментальному определению параметра $\gamma$ в корональной плазме на расстояниях менее 0.1 а.е. немногочисленны. Наряду с подходом Паркера, в качестве примера можно привести работу \cite{2}, в которой с помощью спектрометра EIS на спутнике Hinode наблюдались медленные МГД-волны в корональной петле вблизи поверхности Солнца и было показано, что эффективное значение индекса политропы совпадает с значением Паркера $\gamma_{\text{eff}} = 1.10 \pm 0.02$, причем нагрев плазмы описывался спитцеровской теплопроводностью.
	
	В настоящей работе рассмотрен метод определения индекса политропы $\gamma$ плазмы СВ на участке расширения в короне на высотах от 1 до 5–7 солнечных радиусов по измеряемым ``на месте'' вблизи Земли ионным параметрам СВ: среднему заряду ионов Fe и отношению плотностей ионов кислорода O$^{7+}$/O$^{6+}$. 
	
	Эти ионные параметры сравниваются с модельными ионными параметрами, которые рассчитываются по радиальным профилям температуры и плотности с помощью уравнений баланса для процессов ионизации и рекомбинации ионов в плазме СВ. 
	
	В разделе 2 дается описание модели расчета ионных параметров СВ в короне. В разделе 3 описан метод определения индекса политропы по ионным параметрам для разных типов СВ. В разделе 4 приводятся примеры радиальных распределений индекса политропы $\gamma$ в короне для типовых квазистационарных потоков СВ. В разделе 5 обсуждаются полученные результаты и подводятся заключительные итоги исследования. И, наконец, в Приложении рассмотрены основные особенности и параметры политропной модели.
	
	\section*{2.	МОДЕЛИРОВАНИЕ ИОННЫХ ПАРАМЕТРОВ ПЛАЗМЫ СВ В КОРОНЕ}
	Ионный состав плазмы СВ, т.е. распределение ионов различных элементов по стадиям ионизации, эволюционирует при движении через корону вместе с изменениями физических характеристик плазмы (температуры, плотности и скорости потока) из-за влияния конкурирующих процессов ионизации и рекомбинации в ион-электронных столкновениях. В процессе расширения ионный состав «замораживается» на расстояниях нескольких солнечных радиусов, когда плазма становится бесстолкновительной из-за быстрого падения плотности, и практически не меняется при дальнейшем движении в гелиосфере (см., например, Хундхаузен и др., 1968; Ко и др., 1997; Родькин и др., 2017; Гречнев и др., 2019; Горяев и др., 2020).
	
	Эволюция ионного состава данного химического элемента в движущейся плазменной структуре СВ может быть найдена путем решения уравнений баланса для плотностей ионов $N_Z$ ($Z$ -- заряд соответствующего иона), которые имеют вид
	
\begin{equation}
	\begin{split}
		\frac{dN_Z}{dt} = N_e \big[ & N_{Z-1} C_{Z-1} - N_Z (C_Z + R_{Z-1}) \\
		& + N_{Z+1} R_Z \big],
	\end{split}
	\label{eq:balance}
\end{equation}
	
	где $N_e$ и $T_e$ -- плотность и температура электронов в плазме СВ; $C_Z$ -- скорость ионизации для перехода $N_Z \rightarrow N_{Z+1}$; $R_Z$ -- скорость рекомбинации для перехода $N_{Z+1} \rightarrow N_Z$.
	
	Для расчета зарядового состояния ионов в движущейся плазменной структуре СВ в короне с помощью уравнений баланса (\ref{eq:balance1}) и определения итогового «замороженного» распределения ионов по стадиям ионизации, требуется информация о временных зависимостях параметров плазмы $N_e(t)$, $T_e(t)$, а также скорости потока СВ $V(t)$. Эти параметры зависят от условий нагрева и ускорения, влияющих на состояние плазмы СВ, и от геометрии ее расширения в короне. Соответствующие временные профили обычно моделируются исходя из типичных условий в короне, например, по источникам СВ, либо находятся из измерений с помощью спектроскопических методов (см., например, Гречнев и др., 2019).
	
	В настоящей работе мы рассматриваем квазистационарные условия в выделенном расширяющемся объеме плазмы СВ. В этом случае частная производная в левой части (\ref{eq:balance1}) равна нулю ($\partial N_Z/\partial t = 0$) и при движении плазмы СВ в радиальном направлении уравнение (\ref{eq:balance1}) принимает вид (см., например, Хундхаузен и др., 1968; Ко и др., 1997):
	
	\begin{multline}
		V \frac{dN_Z}{dr} = N_e [N_{Z-1} C_{Z-1}(T_e) - N_Z (C_Z(T_e)\\ + R_{Z-1}(T_e))
		+ N_{Z+1} R_Z(T_e)].
		\label{eq:balance2}
	\end{multline}
	
	В этом случае предполагается, что в каждый момент времени ионный состав плазмы определяется только процессами ионизации и рекомбинации при заданных параметрах плазмы.
	
	Зарядовое состояние ионов в плазме СВ определяется двумя конкурирующими процессами с характерными временными масштабами (см., например, Хундхаузен и др., 1968; Ко и др., 1997). Первый процесс связан с расширением потока СВ в окружающее пространство и характеризуется временем расширения $\tau_{\text{exp}} = N_e/(V dN_e/dr)$. Второй процесс характеризуется временем ионизации/рекомбинации $\tau_{\text{ir}} = 1/(C_Z + R_{Z-1}) N_e$ иона с зарядом $Z$ в плазме СВ. На начальной стадии ускорения и формирования потока СВ вблизи поверхности Солнца, где скорость движения потока мала и плотность плазмы относительно высокая, выполняется условие $\tau_{\text{exp}} \gg \tau_{\text{ir}}$ и ионный состав соответствует ионизационному равновесию при текущей температуре движущейся плазменной структуры.	
		
		\begin{figure*}[p]  % Используйте [p] для размещения на отдельной странице
		\centering
		\includegraphics[width=0.9\linewidth]{images/fig1a.jpg}
	\end{figure*}
	\vspace{1cm}
	\begin{figure*}[p] 
		\centering
		\includegraphics[width=0.9\linewidth]{images/fig1b.jpg}
		
		\caption*{\textbf{Рис 1.} Сравнение времени расширения плазмы $\tau_{\text{exp}}$ с временами ионизации/рекомбинации $\tau_{\text{ir}}$ для ионов O$^{6+}$ – O$^{8+}$ (вверху) и Fe$^{12+}$ – Fe$^{19+}$ (внизу). Для расчета масштабов времен использованы параметры плазмы для КВМ из работ \cite{30} и \cite{12}. Расстояние $r = R/R_{\text{sun}}$ отсчитывается от центра Солнца, т.е. $r = 1$ соответствует поверхности Солнца.}
	\end{figure*}
			
	В процессе дальнейшего движения и расширения увеличивается скорость потока и быстро падает плотность плазмы, в результате чего ионизационное равновесие нарушается. Это проявляется в том, что плотности ионов с меньшими скоростями рекомбинации «замораживаются» раньше и ионный состав отклоняется от равновесного.	
		
	В пределе $\tau_{\text{exp}} \ll \tau_{\text{ir}}$ для всех ионов данного элемента зарядовое состояние плазмы СВ «замораживается» и при дальнейшем движении в межпланетном пространстве практически не меняется.	
	
	На рис. 1 представлено сравнение характерных временных масштабов $\tau_{\text{exp}}$ и $\tau_{\text{ir}}$ для ионов O$^{6+}$ -- O$^{8+}$ и Fe$^{12+}$ -- Fe$^{19+}$, где для расчета этих времен использовались параметры плазмы КВМ из работ Слемзин и др. (2022) и Горяев и др. (2023). Для скоростей рекомбинации и ионизации ионов кислорода и железа использовались атомные данные из базы CHIANTI (Дере и др., 2009). Из рис. 1 можно видеть, что ионный состав элемента O «замораживается» уже на расстояниях порядка двух солнечных радиусов от центра Солнца, тогда как зарядовый состав элемента Fe «замораживается» в среднем на расстояниях 4--8 $R_{\text{sun}}$. Это связано с тем, что скорости рекомбинации для ионов железа заметно превосходят соответствующие скорости для ионов кислорода, что приводит к обратному соотношению для времен рекомбинации этих ионов.
	
	В настоящей работе мы исследуем значения индекса политропы $\gamma$ для различных типов СВ. Для этого решаются уравнения баланса (\ref{eq:balance2}) для ионов элементов O и Fe в рамках подхода, использованного ранее в работах Родькин и др. (2017), Гречнев и др. (2019), Горяев и др. (2020), Слемзин и др. (2022) и Горяев и др. (2023). Мы рассматриваем модель, где плазма СВ расширяется по политропному закону $T_e \propto N_e^{\gamma-1}$ (см. Приложение) из области в короне с максимальной начальной температурой и равновесным распределением ионов при этой температуре. Начальное положение плазмы СВ в момент времени $t = 0$ обычно находится на высотах $\approx 0.1$--0.3 $R_{\text{sun}}$, где достигается максимальная температура и в рамках рассматриваемой модели начинается стадия охлаждения.
		
	Как упоминалось выше, при малых скоростях и высоких плотностях на начальной стадии движения вблизи солнечной поверхности выполняются условия ионизационного равновесия. Отклонения от ионизационного равновесия начинаются выше этих высот, и поэтому конечный «замороженный» ионный состав практически не зависит от выбора начальной точки движения, если она находится вблизи поверхности Солнца. Для радиальных профилей электронной плотности $N_e(r)$ использовалась наиболее вероятная геометрия расширения плазмы СВ в корону: $N_e \propto 1/h^2$ (где $h$ -- высота над поверхностью Солнца) для быстрого и медленного СВ и $N_e \propto 1/r^3$ ($r$ -- расстояние от центра Солнца) для СВ типа КВМ (см., например, Слемзин и др., 2022; Горяев и др., 2023).
	
	Для численного моделирования эволюции ионного состава элементов O и Fe при движении потока СВ от начальной точки до области «замораживания» решались уравнения баланса (\ref{eq:balance2}) с заданными профилями параметров плазмы $N_e(r)$, $T_e(r)$ и $V(r)$. Наконец, по найденным распределения ионов O и Fe определяются ионные параметры $\langle Q_{\text{Fe}} \rangle$ и O$^{7+}$/O$^{6+}$. Меняя индекс политропы, можно получить зависимость параметров $\langle Q_{\text{Fe}} \rangle$ и O$^{7+}$/O$^{6+}$ от $\gamma$. В разделе 3 эта процедура используется для определения параметра $\gamma$ для разных типов СВ с помощью сравнения рассчитанных ионных параметров $\langle Q_{\text{Fe}} \rangle$ и O$^{7+}$/O$^{6+}$ с измеренными распределениями.
	
	\begin{figure*}[p]
		\centering
		\includegraphics[width=0.70\linewidth]{images/fig2ab.jpg}
		
		\vspace{0.5cm}
	\end{figure*}
	
	\begin{figure*}[p]
		\centering
		\includegraphics[width=0.70\linewidth]{images/fig2cd.jpg}
		
		\caption*{\textbf{Рис 2.} Гистограммы ионных параметров $\langle Q_{\mathrm{Fe}} \rangle$ (a) и $\mathrm{O}^{7+}/\mathrm{O}^{6+}$ (c) для потоков медленного СВ (SWslow), (b) и (d) -- модельные зависимости ионных параметров от индекса $\gamma$. Пунктирные линии соответствуют средним значениям параметров, точечные -- отклонениям от среднего на величину дисперсии параметра при построении гистограммы. В заголовках указаны типы параметров и общее количество часовых интервалов данного типа СВ по каталогу ИКИ (см. текст) за 2010 г.}
		
	\end{figure*}
		
\section*{3. ОПРЕДЕЛЕНИЕ ИНДЕКСА ПОЛИТРОПЫ ДЛЯ РАЗНЫХ ТИПОВ СВ ПО ИОННЫМ ПАРАМЕТРАМ $\langle Q_{\text{Fe}} \rangle$ и O$^{7+}$/O$^{6+}$}
Для анализа статистики ионных параметров плазмы СВ, относящихся к разным крупномасштабным типам СВ и регистрируемых на расстоянии 1 а.е., были построены гистограммы среднего заряда ионов железа $\langle Q_{\text{Fe}} \rangle$ и отношения ионов кислорода O$^{7+}$/O$^{6+}$ и по ним получены средние значения этих параметров и их дисперсии. Для этого использовались данные прибора SWICS на космическом аппарате ACE (Глоеклер и др., 1998) за 2010 г., разграниченные по времени по типам потоков СВ в соответствии с каталогом ИКИ (\textbackslash{}url\{http://www.iki.rssi.ru/omni/catalog/\}). Этот период соответствует спокойной солнечной короне в начальной стадии роста активности 24-го солнечного цикла. Далее в этом разделе приводятся гистограммы ионных параметров $\langle Q_{\text{Fe}} \rangle$ и O$^{7+}$/O$^{6+}$ для квазистационарных потоков по классификации каталога ИКИ: медленного СВ (SW\textsubscript{slow}), быстрого СВ (SW\textsubscript{fast}, CIR) и нестационарных потоков КВМ типа магнитных облаков (MC -- magnetic clouds) и EJECTA.

Эти эмпирические данные сравниваются с модельными расчетами ионных параметров $\langle Q_{\text{Fe}} \rangle$ и O$^{7+}$/O$^{6+}$ как функций индекса $\gamma$, где по средним значениям и дисперсиям этих параметров определяются средние величины $\gamma$ и их разброс для разных типов СВ. При моделировании ионных параметров с помощью уравнений (2) для физических характеристик плазмы СВ (плотности, температуры и скорости потока) использовались данные из работ Ко и др. (1997) (быстрый СВ) и Слемзин и др. (2022) (медленный СВ и КВМ). В работе Ко и др. (1997) исследовались физические условия в полярной корональной дыре. В работе Слемзин и др. (2022) рассматривалось событие 18 августа 2010 г., связанное с формированием КВМ и постэруптивным потоком СВ.

\subsection*{3.1 Медленный СВ (SW\textsubscript{slow})}

На рис. 2 представлены гистограммы ионных параметров $\langle Q_{\text{Fe}} \rangle$ и O$^{7+}$/O$^{6+}$ (верхние рисунки) для потоков медленного СВ (SW\textsubscript{slow}). На нижних рисунках приведены соответствующие модельные расчеты зависимости индекса $\gamma$ от ионных параметров. При моделировании ионных параметров в качестве исходной информации для плотности, температуры и скорости потока использовались данные из работы Слемзин и др. (2022), соответствующие временным интервалам T1 (до-эруптивный поток СВ 19.08.2010 г. - 20.08.2010 г.) и T4 (вернувшийся в доэруптивное состояние поток СВ 23.08.2010 г. -- 25.08.2010 г.).

\subsection*{3.2 Быстрый СВ (SW\textsubscript{fast}, CIR)}

Гистограммы для быстрого СВ представлены на рис. 3 (SW\textsubscript{fast}) и рис. 4 (CIR). При моделировании ионных параметров быстрого СВ, как типа SW\textsubscript{fast}, так и CIR, в качестве исходных данных использовались радиальные профили плотности, температуры и скорости потока для СВ из полярных областей Солнца, приведенные в работе Ко и др. (1997). Соответствующие расчетные зависимости индекса $\gamma$ от ионных параметров $\langle Q_{\text{Fe}} \rangle$ и O$^{7+}$/O$^{6+}$ представлены в нижних частях рис. 3 и 4.

\begin{figure*}[p]
	\centering
	\includegraphics[width=0.70\linewidth]{images/fig3ab.jpg}
	\vspace{0.5cm}
\end{figure*}
\begin{figure*}[p]
	\centering
	\includegraphics[width=0.70\linewidth]{images/fig3cd.jpg}
	
	\caption*{\textbf{Рис. 3.}Гистограммы ионных параметров и модельные значения $\gamma$ для быстрого СВ (SWfast).  Гистограммы ионных параметров $\langle Q_{\mathrm{Fe}} \rangle$ (a) и $\mathrm{O}^{7+}/\mathrm{O}^{6+}$ (c) для потоков медленного СВ (SWslow), (b) и (d) -- модельные зависимости ионных параметров от индекса $\gamma$. Пунктирные линии соответствуют средним значениям параметров, точечные -- отклонениям от среднего на величину дисперсии параметра при построении гистограммы. В заголовках указаны типы параметров и общее количество часовых интервалов данного типа СВ по каталогу ИКИ (см. текст) за 2010 г.}
\end{figure*}

\begin{figure*}[p]
	\centering
	\includegraphics[width=0.70\linewidth]{images/fig4ab.jpg}
	
\end{figure*}
\vspace{1cm}
\begin{figure*}[p]
	\centering
	\includegraphics[width=0.70\linewidth]{images/fig4cd.jpg}
	\caption*{\textbf{ Рис. 4.} Гистограммы ионных параметров и модельные значения $\gamma$ для рекуррентных потоков СВ (CIR).  Гистограммы ионных параметров <QFe> (a)	 и O7+/O6+ (c) для потоков медленного СВ (SWslow), (b) и (d) – модельные зависимости ионных параметров от $\gamma$. Пунктирные линии соответствуют средним значениям параметров, точечные – отклонениям от среднего на величину дисперсии параметра при построении гистограммы. В заголовках указаны типы параметров и общее количество часовых интервалов данного типа СВ по каталогу ИКИ (см. текст) за 2010 г..}
	
\end{figure*}

\subsection*{3.3 МКВМ (MC, EJECTA)}

На рис. 5 и 6 представлены гистограммы ионных параметров $\langle Q_{\text{Fe}} \rangle$ и O$^{7+}$/O$^{6+}$ для МКВМ типа MC (рис. 5) и EJECTA (рис. 6). При моделировании ионных параметров МКВМ типа MC в качестве исходных данных для физических характеристик плазмы СВ использовались результаты работы Слемзин и др. (2022) для временного интервала T2 (эрупция, соответствующая распространению МКВМ 20.08.2010 г. -- 21.08.2010 г.), а при моделировании потоков типа EJECTA -- из той же работы Слемзин и др. (2022) для интервала T3 (пост-эруптивный поток СВ 21.08.2010 г. -- 22.08.2010 г.).
В работе Слемзин и др. (2022) интервал T3 по соотношению «холодной» (относящейся к низкозарядным ионам в распределении по ионному заряду Z) и «горячей» (относящейся к высокозарядным ионам в распределении по Z) частей зарядового распределения ионов Fe в плазме СВ (в Слемзин и др. (2022) они обозначены параметрами q4 и q12 соответственно) рассматривается как пост-эруптивный поток, но по общим характеристикам плазмы этот интервал можно отнести к типу EJECTA.

Полученные из гистограмм на рис. 2--6 средние значения и дисперсии ионных параметров $\langle Q_{\text{Fe}} \rangle$ и O$^{7+}$/O$^{6+}$, а также средние значения и дисперсии значений индекса $\gamma$, найденные при сопоставлении распределений на гистограммах с результатами моделирования приведены в табл. 1. Отметим, что в табл. 1 не указаны границы значений $\gamma$, соответствующие отклонению параметра O$^{7+}$/O$^{6+}$ от среднего на величину дисперсии, поскольку они выходят за границы применимости данной модели.

	\section*{4.	РАДИАЛЬНЫЕ РАСПРЕДЕЛЕНИЯ ИНДЕКСА ПОЛИТРОПЫ В КОРОНЕ ДЛЯ КВАЗИСТАЦИОНАРНЫХ ПОТОКОВ СВ}
	Если из измерений или других данных известны распределения температуры и плотности плазмы СВ в короне, то по ним также можно найти распределение индекса γ в рамках политропной модели (см. формулу (П2) в приложении). На рис. 7 показаны примеры радиальных распределений индекса γ для квазистационарных потоков СВ, полученных по данным модельных расчетов и из измерений температуры и плотности плазмы в короне при низкой солнечной активности в работах Кранмер и др. (2007), Ланди и др. (2012), Ко и др. (1997), Гибсон и др. (1999), Горяев и др. (2014).  Распределение индекса γ на рис. 7(а) соответствует профилям характеристик плазмы из работ Кранмер и др. (2007) и Ланди и др. (2012) для медленного СВ. На рис. 7(b) представлено распределение γ для медленного СВ из стримеров по данным работ Гибсон и др. (1999), Горяев и др. (2014). И наконец, рис. 7(c) демонстрирует профили индекса политропы для быстрого СВ из полярных корональных дыр, полученные по данным работ Ко и др. (1997), Фишер, Гухатакурта (1995). Для сравнения с значениями индекса политропы, полученными из распределений ионных параметров согласно таблице 1, радиальные профили γ усреднялись и соответствующие результаты приведены в таблице 2.
	
	\section*{5.	ОБСУЖДЕНИЕ РЕЗУЛЬТАТОВ И ВЫВОДЫ}
		
	В настоящей работе по результатам моделирования ионных параметров СВ и их сопоставления с измеренными значениями зарядового состояния ионов железа $\langle Q_{\mathrm{Fe}} \rangle$ и кислорода O$^{7+}$/O$^{6+}$ получены «на месте» усредненные значения индекса политропы крупномасштабных потоков СВ для спокойной короны (2010 г.) на высотах $\approx 1$--7 $R_{\mathrm{sun}}$.
	
	Отметим следующие особенности полученных результатов. Для обоих ионных параметров большие значения индекса $\gamma$ соответствуют меньшим значениям параметров СВ, т.е. более низким температурам «замораживания» ионного состава и наоборот.
	
	Для квазистационарных потоков типа медленного СВ (SW\textsubscript{slow}) и высокоскоростного СВ (SW\textsubscript{fast}) средние значения $\gamma$, найденные из анализа гистограмм ионов железа и кислорода, практически совпадают, что говорит об их формировании на малых высотах от поверхности Солнца $\lesssim 1 R_{\mathrm{sun}}$.
	
	\begin{figure*}[p]
		\centering
		\includegraphics[width=0.70\linewidth]{images/fig5ab.jpg}
		
	\end{figure*}
	\vspace{1cm}
	
	\begin{figure*}[p]
		\centering
		\includegraphics[width=0.70\linewidth]{images/fig5cd.jpg}
		\caption*{\textbf{Рис 5.} Гистограммы ионных параметров и модельные значения $\gamma$ для МКВМ типа магнитных облаков (MC). Гистограммы ионных параметров <QFe> (a)	 и O7+/O6+ (c) для потоков медленного СВ (SWslow), (b) и (d) – модельные зависимости ионных параметров от $\gamma$. Пунктирные линии соответствуют средним значениям параметров, точечные – отклонениям от среднего на величину дисперсии параметра при построении гистограммы. В заголовках указаны типы параметров и общее количество часовых интервалов данного типа СВ по каталогу ИКИ (см. текст) за 2010 г.}
	\end{figure*}
		
	\begin{figure*}[p]
		\centering
		\includegraphics[width=0.70\linewidth]{images/fig6ab.jpg}
		
	\end{figure*}
	
	\begin{figure*}[p]
		\centering
		\includegraphics[width=0.70\linewidth]{images/fig6cd.jpg}
		\caption*{\textbf{Рис 6.} Гистограммы ионных параметров и модельные значения $\gamma$ для МКВМ типа EJECTA. Гистограммы ионных параметров <QFe> (a)	 и O7+/O6+ (c) для потоков медленного СВ (SWslow), (b) и (d) – модельные зависимости ионных параметров от $\gamma$. Пунктирные линии соответствуют средним значениям параметров, точечные – отклонениям от среднего на величину дисперсии параметра при построении гистограммы. В заголовках указаны типы параметров и общее количество часовых интервалов данного типа СВ по каталогу ИКИ (см. текст) за 2010 г..}
		
	\end{figure*}
	
	Основные максимумы гистограмм для высокоскоростных потоков типа SW\textsubscript{fast} и CIR соответствуют экваториальным КД. Предположительно, наличие второго (меньшего по высоте) максимума в гистограммах связано с тем, что в некоторые периоды в центральной части солнечного диска, помимо экваториальных КД, присутствовали низкоширотные участки полярных КД, для которых поток СВ имеет более низкую температуру. В 2010 г. наблюдалось 49 высокоскоростных потоков СВ, из которых в 32 событиях источниками были только экваториальные корональные дыры, а в 17 наблюдались низкоширотные области полярных дыр. Однако в нашем случае эти особенности не влияют заметно на результат определения индекса политропы по используемым зарядовым параметрам, поскольку различия в средних значениях параметров $\langle Q_{\mathrm{Fe}} \rangle$ и O$^{7+}$/O$^{6+}$ для СВ из экваториальных и полярных корональных дыр находятся в пределах их дисперсий.
	
	Для нестационарных потоков СВ типа CIR, MC, EJECTA значения $\gamma$, полученные по отношению ионов кислорода, отличаются от значений, найденных по ионам железа, и имеют большую дисперсию. Это означает, что термодинамика потоков СВ на промежуточных высотах от 1 до 2 $R_{\mathrm{sun}}$ и на больших высотах различна. Для CIR и EJECTA значения индексов $\gamma$, полученные по отношению O$^{7+}$/O$^{6+}$, меньше 1, что указывает на то, что плазма еще находится в стадии нагрева. Для MC значение $\gamma > 2$, что указывает на существование дополнительного канала расхода тепловой энергии сверх охлаждения в ходе адиабатического расширения.
	\begin{figure*}[p]
		\centering
		
		\includegraphics[width=0.90\linewidth]{images/fig7a.jpg}
		
	\end{figure*}
	
	\begin{figure*}[htbp]
		\centering
		\includegraphics[width=0.90\linewidth]{images/fig7b.jpg}
		
	\end{figure*}
	
	\begin{figure*}[htbp]
		\centering
		\includegraphics[width=0.90\linewidth]{images/fig7c.jpg}
		\caption*{\textbf{Рис. 7.} Значения $\gamma$, полученные из радиальных профилей температуры и плотности для разных типов СВ из работ Кранмер и др. (2007), Ланди и др. (2012), Гибсон и др. (1999), Горяев и др. (2014), Ко и др. (1997), Фишер, Гухатакурта (1995). (a) медленный СВ (Кранмер и др., 2007; Ланди и др., 2012); (b) медленный СВ из стримеров: сплошная линия по данным Гибсон и др. (1999), точки с ошибками измерений по данным Горяев и др. (2014); (c) быстрый СВ из полярных корональных дыр: северной 1N и южной 1S по данным Фишер, Гухатакурта (1995) и северной 2N по данным Ко и др. (1997). }
		
	\end{figure*} 
	И наконец, рис. 7(c) демонстрирует профили индекса политропы для быстрого СВ из полярных корональных дыр, полученные по данным работ \cite{16,32}. Для сравнения с значениями индекса политропы, полученными из распределений ионных параметров согласно таблице 1, радиальные профили $\gamma$ усреднялись и соответствующие результаты приведены в таблице 2.
	Из приведенных результатов можно сделать вывод, что формирование температурного режима квазистационарных потоков медленного и быстрого СВ происходит ниже высот 1--1.5 $R_{\mathrm{sun}}$, где зарядовый состав ионов кислорода «замораживается». Для нестационарных потоков режим политропы, связанный с расширением и охлаждением плазмы, формируется выше этого уровня, но ниже высот $\approx 5$--7 $R_{\mathrm{sun}}$, на которых происходит «замораживание» зарядового состава ионов железа.
	
	В целом, значения индекса $\gamma$, рассчитываемые по средним значениям ионных параметров методом моделирования варьируемых радиальных распределений температуры, согласуются в пределах дисперсии ионных параметров с значениями этого индекса, полученными из радиальных профилей температуры и плотности. Результаты работы показывают возможность исследования термодинамики СВ в короне по данным об ионном составе плазмы СВ.
	
	\begin{table*}[htbp]
		\centering
		\normalsize 
		\caption{Средние значения и дисперсия ионных параметров СВ по данным ACE/SWICS за 2010 г.}
		\label{tab:ion_parameters}
		\begin{tabular}{|l|l|c|c|}
			\hline
			\textbf{Тип СВ} & \textbf{Ионный параметр} & \makecell{\textbf{Среднее} \\ \textbf{и дисперсия}} & \makecell{\textbf{Индекс} \\ $\gamma$} \\
			\hline
			\multirow{2}{*}{SWslow} 
			& $\langle Q_{\text{Fe}} \rangle$ & $9.70 \pm 0.57$ & $1.03 \pm 0.02$ \\
			\cline{2-4}
			& O$^{7+}$/O$^{6+}$ & $0.156 \pm 0.114$ & $1.03$ \\
			\hline
			\multirow{2}{*}{SWfast}
			& $\langle Q_{\text{Fe}} \rangle$ & $9.42 \pm 0.44$ & $1.14 \pm 0.04$ \\
			\cline{2-4}
			& O$^{7+}$/O$^{6+}$ & $0.0446 \pm 0.0446$ & $1.14$ \\
			\hline
			\multirow{2}{*}{CIR}
			& $\langle Q_{\text{Fe}} \rangle$ & $9.47 \pm 0.46$ & $1.13 \pm 0.05$ \\
			\cline{2-4}
			& O$^{7+}$/O$^{6+}$ & $0.125 \pm 0.112$ & $0.77$ \\
			\hline
			\multirow{2}{*}{MC}
			& $\langle Q_{\text{Fe}} \rangle$ & $11.37 \pm 1.22$ & $1.21 \pm 0.03$ \\
			\cline{2-4}
			& O$^{7+}$/O$^{6+}$ & $0.475 \pm 0.256$ & $2.10$ \\
			\hline
			\multirow{2}{*}{EJECTA}
			& $\langle Q_{\text{Fe}} \rangle$ & $9.89 \pm 0.67$ & $1.21 \pm 0.02$ \\
			\cline{2-4}
			& O$^{7+}$/O$^{6+}$ & $0.177 \pm 0.127$ & $0.90$ \\
			\hline
		\end{tabular}
	\end{table*}
	
	\begin{table*}[htbp]
		\centering
		\normalsize 
		\caption{Значения индекса $\gamma$ для медленного и быстрого СВ, усредненные по радиальным распределениям индекса.}
		\label{tab:gamma_radial}
		\begin{tabular}{|l|c|c|c|}
			\hline
			\textbf{Тип СВ} & \textbf{Диапазон, $R_{\text{sun}}$} & \textbf{Среднее значение $\gamma$} & \textbf{Исходные данные} \\
			\hline
			Медленный & 1.5–5 & $1.076 \pm 0.010$ & \cite{17}, \cite{19} \\
			\hline
			Медленный (из стримера) & 1.45–2 & $1.043 \pm 0.026$ & \cite{10} \\
			\hline
			Медленный & 1.5–2.2 & $1.075 \pm 0.024$ & \cite{8} \\
			\hline
			Быстрый (из северной ПКД) & 2–6 & $1.196 \pm 0.034$ & \cite{16} \\
			\hline
			Быстрый (из северной ПКД) & 2–5.5 & $1.183 \pm 0.042$ & \cite{32} \\
			\hline
			Быстрый (из южной ПКД) & 1.9–5.5 & $1.167 \pm 0.024$ & \cite{32} \\
			\hline
		\end{tabular}
	\end{table*}
	
	\section*{
		ПРИЛОЖЕНИЕ: ПОЛИТРОПНАЯ МОДЕЛЬ СВ В КОРОНЕ СОЛНЦА
	}
	
	
	Основным параметром политропной модели является индекс $\gamma$, который можно выразить через теплоемкости при постоянном давлении $c_p$, постоянном объеме $c_v$ и теплоемкость политропического процесса $c$ (все теплоемкости в пересчете на единичную плотность):
	
	\begin{equation}
		\gamma = \frac{c_p - c}{c_v - c}.
		\tag{П1}
	\end{equation}
	
	При $c = 0$, что соответствует адиабатическому процессу, имеем показатель адиабаты $\gamma = c_p/c_v = 5/3 = \alpha$. В случае потока плазмы СВ в короне удобнее оперировать переменными температуры $T$ и плотности $n$, в которых уравнение политропы имеет вид
	
	\begin{equation}
		T n^{1-\gamma} = \text{const} \equiv C.
		\tag{П2}
	\end{equation}
	
	Из измерений радиальных профилей температуры и плотности параметр политропы может быть получен с помощью уравнения:
	
	\begin{equation}
		\ln(T) = (\gamma - 1) \ln(n) + \ln(C).
		\tag{П3}
	\end{equation}
	
	Дифференцируя (\ref{eq:poly_log}) по высоте $r$, получим зависимость между радиальными градиентами температуры и плотности:
	
	\begin{equation}
		\frac{1}{T} \frac{dT}{dr} = (\gamma - 1) \frac{1}{n} \frac{dn}{dr},
		\tag{П4}
	\end{equation}
	
	где $r$ -- высота от поверхности Солнца. Если положить, что температура зависит от высоты по степенному закону $T \propto r^{-\delta}$, а плотность $n \propto r^{-\beta}$, то для индекса политропы получаем соотношение (Тоттен и др., 1995):
	
	\begin{equation}
		\gamma = 1 + \frac{\delta}{\beta}.
		\tag{П5}
	\end{equation}
	
	Согласно модели Паркера (см.\cite{20,21}), в нижней короне при относительно высокой плотности плазма находится в термодинамическом равновесии с окружающей средой при постоянной температуре, чему соответствует индекс политропы $\gamma = 1$. Значение $\gamma < 1$ означает, что при расширении в корону температура плазмы растет, в то время как плотность падает. Этот режим описывает начальную «нагревную» стадию формирования СВ до максимальной температуры.
	
	В другом пределе, который отвечает режиму сверхзвукового течения в гелиосфере при низкой плотности, отсутствует взаимодействие плазмы СВ с окружающей средой, когда невозмущенная стационарно истекающая плазма расширяется и охлаждается адиабатически с $\gamma = \alpha = 5/3$. В средней короне значения индекса политропы, удовлетворяющие условию $1 < \gamma < \alpha$, соответствуют расширению плазмы СВ с одновременным охлаждением в присутствии некоторого притока энергии, который уменьшает снижение температуры по сравнению с адиабатическим режимом. Значения индекса $\gamma > \alpha$ означают, что температура плазмы с падением плотности уменьшается сильнее, чем это происходит в адиабатическом режиме, то есть помимо работы на расширение в плазме СВ имеется дополнительный канал диссипации тепловой энергии.
	
	Зная показатель политропы и радиальные зависимости температуры и плотности, поток энергии $\epsilon$ (на единицу объема), нагревающий расширяющуюся в короне плазму при изменении температуры на величину $dT$ на участке $dr$ за время $dt$, можно оценить следующим образом:
	
	\begin{equation}
		\begin{split}
			\epsilon &= c n \frac{dT}{dt} k_B = \frac{3}{2} \frac{\gamma - \alpha}{\gamma - 1} \frac{dT}{dt} n k_B \\
			&= \frac{3}{2} \frac{\gamma - \alpha}{\gamma - 1} \frac{dT}{dr} V n k_B \\
			&= \frac{3}{2} (\gamma - \alpha) \frac{dn}{dr} V n k_B,
		\end{split}
		\tag{П6}
	\end{equation}
	
	где $k_B$ -- постоянная Больцмана, $V$ -- скорость потока.
	
	Согласно Тоттену и др. (1995), для учета влияния магнитного поля в уравнение политропы к газовому давлению $p_{\text{г}} = n k_B T$ нужно добавить магнитное давление $p_{\text{м}} = B^2/(2\mu_0)$:
	
	\begin{equation}
		(p_{\text{г}} + p_{\text{м}}) n^{-\gamma} = k_B T \left(1 + \frac{1}{\beta_{\text{пл}}}\right) n^{1-\gamma} = \text{const},
		\tag{П7}
	\end{equation}
	
	где $\beta_{\text{пл}} = p_{\text{г}}/p_{\text{м}}$ -- плазменная бета. Если радиальная функция $K_{\text{м}} = 1 + 1/\beta_{\text{пл}}$ в некотором диапазоне высот $r$ может быть представлена как $K_{\text{м}} \sim r^{\mu}$, то с учетом магнитного давления в соответствии с (\ref{eq:poly_indices}) получим:
	
	\begin{equation}
		\gamma = \gamma_{\text{г}} - \frac{\mu}{\beta},
		\tag{П8}
	\end{equation}
	
	где $\gamma_{\text{г}}$ -- индекс, соответствующий только газовому давлению. В настоящей работе мы определяем индекс политропы по измеренным ионным параметрам СВ, которые на основе моделирования связаны с реальными радиальными распределениями температуры и плотности, в результате чего влияние магнитного давления учитывается автоматически.
	
	Наконец, в отношении политропного соотношения (\ref{eq:poly_eq}) следует отметить, что плотность корональной плазмы складывается из плотности протонов и электронов, т.е. $n \approx N_p + N_e$. В то же время, как указано в разделе 2, при анализе ионного состава используются плотность и температура электронов. С достаточной для исследуемой модели точностью можно считать, что в пределах рассматриваемых высот до $\approx 4$--5 $R_{\text{sun}}$ (где $R_{\text{sun}}$ -- радиус Солнца) плазма СВ находится в термодинамическом равновесии, что отражается в приближенном равенстве $T_p \approx T_e$ (где $T_p$ и $T_e$ -- температура протонов и электронов, соответственно), а плотности протонов и электронов даются соотношениями $N_p \approx N_e$ и $n \approx 2N_e$. Из уравнения (\ref{eq:poly_log}) следует, что удвоение плотности, так же как и кратное изменение температуры при сохранении формы радиальных профилей, не влияют на значение индекса политропы, поэтому приведенные выше соотношения сохраняются при замене переменных $(n, T)$ на $(N_e, T_e)$.
	
	\section*{БЛАГОДАРНОСТИ}
	Авторы выражают благодарность команде прибора ACE/SWICS и сотрудникам ACE Science Center за предоставление данных об ионном составе солнечного ветра, а также Ю.И. Ермолаеву и сотрудникам Лаборатории солнечного ветра ИКИ РАН за предоставление данных каталога крупномасштабных структур солнечного ветра.
	
	\begin{thebibliography}{99}
		\bibitem{1} Ашванден М., Физика солнечной короны [Aschwanden M.J., \textit{Physics of the Solar Corona}. -- Springer Berlin, Heidelberg, 2005], гл. 17.
		\bibitem{2} Ван Доорсселаере и др. (T. Van Doorsselaere, N. Wardle, G. Del Zanna, K. Jansari, E. Verwichte, and V.M. Nakariakov) // \textit{Astrophys. J. Lett.} -- 2011. -- Vol. 727. -- P. L32.
		\bibitem{3} Гибсон и др. (S.E. Gibson, A. Fludra, F. Bagenal, D. Biesecker, G. del Zanna, and B. Bromage) // \textit{J. Geophys. Res.} -- 1999. -- Vol. 104. -- P. 9691--9700.
		\bibitem{4} Глоеклер и др. (G. Gloeckler, J. Cain, F.M. Ipavich, E.O. Tums, P. Bedini, L.A. Fisk, T.H. Zurbuchen, P. Bochsler, et al.) // \textit{Space Sci. Rev.} -- 1998. -- Vol. 86. -- P. 497--539.
		\bibitem{5} Горяев и др. (F. Goryaev, V. Slemzin, L. Vainshtein, and D.R. Williams) // \textit{Astrophys. J.} -- 2014. -- Vol. 781. -- P. 100.
		\bibitem{6} Горяев и др. (F.F. Goryaev, V. Slemzin, and D. Rodkin) // \textit{Astrophys. J. Lett.} -- 2020. -- Vol. 905. -- P. L17.
		\bibitem{7} Горяев Ф.Ф., Слемзин В.А., Родькин Д.Г., Шугай Ю.С. // \textit{Космич. исслед.} -- 2023. -- Т. 61. -- С. 10--21. [F.F. Goryaev, V.A. Slemzin, D.G. Rodkin, and Yu.S. Shugai // \textit{Cosmic Res.} -- 2023. -- Vol. 61. -- P. 8--19.]
		\bibitem{8} Гречнев и др. (V.V. Grechnev, A.A. Kochanov, A.M. Uralov, V.A. Slemzin, D.G. Rodkin, F.F. Goryaev, V.I. Kiselev, and I.I. Myshyakov) // \textit{Solar Phys.} -- 2019. -- Vol. 294. -- P. 139.
		\bibitem{9} Дере и др. (K.P. Dere, E. Landi, P.R. Young, G. Del Zanna, M. Landini, and H.E. Mason) // \textit{Astron. Astrophys.} -- 2009. -- Vol. 498. -- P. 915--929.
		\bibitem{10} Зеленый Л.М., Веселовский И.С. (Ред.) \textit{Плазменная гелиогеофизика}. -- М.: Физматлит, 2008. -- Т. 1. -- Гл. 3.
		\bibitem{11} Ко и др. (Y.-K. Ko, L.A. Fisk, J. Geiss, G. Gloeckler, and M. Guhathakurta) // \textit{Solar Phys.} -- 1997. -- Vol. 171. -- P. 345--361.
		\bibitem{12} Кранмер и др. (S.R. Cranmer, A.A. van Ballegooijen, and R.J. Edgar) // \textit{Astrophys. J. Suppl. Ser.} -- 2007. -- Vol. 171. -- P. 520--551.
		\bibitem{13} Кутузов А.С., Чашей И.В. // \textit{Геомагнетизм и аэрон.} -- 1998. -- Т. 38. -- С. 1--7. [A.S. Kutuzov, I.V. Chashei // \textit{Geomagnetism and Aeron.} -- 1998. -- Vol. 38. -- P. 139--145.]
		\bibitem{14} Ланди и др. (E. Landi, J.R. Gruesbeck, S.T. Lepri, and T.H. Zurbuchen) // \textit{Astrophys. J.} -- 2012. -- Vol. 750. -- P. 159.
		\bibitem{15} Ливадиотис (G. Livadiotis) // \textit{Entropy} -- 2018. -- Vol. 20. -- P. 799.
		\bibitem{16} Лловерас и др. (D.G. Lloveras, A.M. Vásquez, F.A. Nuevo, C. Mac Cormack, N. Sachdeva, W. Manchester, B. Van der Holst, and R.A. Frazin) // \textit{Solar Phys.} -- 2020. -- Vol. 295. -- P. 76.
		\bibitem{17} Майс и др. (M.L. Mays, A. Taktakishvili, A. Pulkkinen, P.J. MacNeice, L. Rastatter, D. Odstrcil, L.K. Jian, I.G. Richardson, et al.) // \textit{Solar Phys.} -- 2015. -- Vol. 290. -- P. 1775--1810.
		\bibitem{18} Мозер и др. (F.S. Mozer, O.V. Agapitov, J.C. Kasper, R. Livi, O. Romeo, and I.Y. Vasko) // \textit{Astron. Astrophys.} -- 2023. -- Vol. 673. -- P. L3.
		\bibitem{19} Николау и др. (G. Nicolaou, G. Livadiotis, R.T. Wicks, D. Verscharen, and B.A. Maruca) // \textit{Astrophys. J.} -- 2020. -- Vol. 901. -- P. 26.
		\bibitem{20} Паркер (E.N. Parker) // \textit{Astrophys. J.} -- 1958. -- Vol. 128. -- P. 664--676.
		\bibitem{21} Паркер Е.Н. \textit{Динамические процессы в межпланетной среде}. -- М.: Мир, 1965.
		\bibitem{22} Перроне и др. (D. Perrone, S. Perri, R. Bruno, D. Stansby, R. D'Amicis, V.K. Jagarlamudi, R. Laker, S. Toledo-Redondo, et al.) // \textit{Astron. Astrophys.} -- 2022. -- Vol. 668. -- P. A189.
		\bibitem{23} Помоелл, Поедтс (J. Pomoell and S. Poedts) // \textit{J. Space Weather and Space Clim.} -- 2018. -- Vol. 8. -- P. A35.
		\bibitem{24} Родькин и др. (D. Rodkin, F. Goryaev, P. Pagano, G. Gibb, V. Slemzin, Y. Shugay, I. Veselovsky, and D.H. Mackay) // \textit{Solar Phys.} -- 2017. -- Vol. 292. -- P. 90.
		\bibitem{25} Слемзин и др. (V. Slemzin, F. Goryaev, and D. Rodkin) // \textit{Astrophys. J.} -- 2022. -- Vol. 929. -- P. 146.
		\bibitem{26} Тоттен и др. (T.L. Totten, J.W. Freeman, and S. Arya) // \textit{J. Geophys. Res.} -- 1995. -- Vol. 100. -- P. 13--17.
		\bibitem{27} Фишер, Гухатакурта (R. Fisher and M. Guhathakurta) // \textit{Astrophys. J.} -- 1995. -- Vol. 447. -- P. L139--L142.
		\bibitem{28} Хундхаузен и др. (A.J. Hundhausen, H.E. Gilbert, and S.J. Bame) // \textit{Astrophys. J. Lett.} -- 1968. -- Vol. 152. -- P. L3--L5.
		\bibitem{29} Хундхаузен А. \textit{Расширение короны и солнечный ветер}. -- М.: Мир, 1976. -- Гл. 3.
		\bibitem{30} Якобс, Поедтс (C. Jacobs and S. Poedts) // \textit{Adv. Space Res.} -- 2011. -- Vol. 48. -- P. 1958--1966.
	\end{thebibliography}
	 % файл references.bib
	
\end{document}