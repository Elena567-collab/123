% !TeX program = xelatex
% !TeX encoding = UTF-8
\documentclass[a4paper,12pt]{article}

% ===== ПАКЕТЫ =====
\usepackage{geometry}
\geometry{left=2.5cm, right=2cm, top=2cm, bottom=2cm}
\usepackage{graphicx}
\usepackage{array}
\usepackage{multirow}
\usepackage{longtable}
\usepackage{amsmath}
\usepackage{caption}
\usepackage{setspace}
\usepackage{xcolor}
\usepackage[table]{xcolor}
\usepackage{float} % для [H] размещения

% ===== ЯЗЫК И ШРИФТЫ =====
\usepackage{polyglossia}
\setmainlanguage{russian}
\setotherlanguage{english}
\usepackage{fontspec}
\setmainfont{SourceSerifPro}[
Path = ./fonts/,
Extension = .ttf,
UprightFont = *-Regular,
BoldFont = *-Bold,
ItalicFont = *-Italic,
BoldItalicFont = *-BoldItalic
]

% ===== ЦВЕТА =====
\definecolor{headerblue}{RGB}{62, 146, 199}
\definecolor{lightblue}{RGB}{222, 234, 246}
\definecolor{orange}{RGB}{255, 165, 0}

% ===== НАСТРОЙКИ =====
\onehalfspacing
\sloppy
\emergencystretch=3em

% Настройки разрывов
\raggedbottom
\widowpenalty=10000
\clubpenalty=10000
\interlinepenalty=100
\brokenpenalty=10000

% ===== ПРОСТЫЕ КОМАНДЫ БЕЗ \leaders =====
\newcommand{\simpleheader}[2]{%
	\noindent%
	\colorbox{#1}{%
		\parbox{\dimexpr\textwidth-2\fboxsep}{%
			\centering\color{white}\textbf{#2}%
		}%
	}%
	\par\vspace{10pt}%
}

% ===== НАЧАЛО ДОКУМЕНТА =====
\begin{document}
	
	% ========== ПЕРВЫЙ ОТБОРОЧНЫЙ ЭТАП ==========
	\begin{center}
		{\Large\bfseries ПЕРВЫЙ ОТБОРОЧНЫЙ ЭТАП}
		
		\vspace{10pt}
		
		{\large\bfseries ПРЕДМЕТНЫЙ ТУР}
	\end{center}
	
	\vspace{20pt}
	
	% ========== ХИМИЯ ==========
	\noindent{\fontsize{14}{16.8}\selectfont\bfseries Химия}\par\vspace{10pt}
	
	% ЗАГОЛОВОК С СИНИМ ФОНОМ И БЕЛЫМ ТЕКСТОМ
	\noindent
	\colorbox{headerblue}{%
		\parbox{\dimexpr\textwidth-2\fboxsep}{%
			\vspace{4pt}%
			\centering\color{white}\large\textbf{Задача 1 \hfill 30 баллов}%
			\vspace{4pt}%
		}%
	}
	
	% ТЕМА С ГОЛУБЫМ ФОНОМ
	\noindent
	\colorbox{lightblue}{%
		\parbox{\dimexpr\textwidth-2\fboxsep}{%
			\vspace{5pt}%
			{\color{black}\small\textit{Tема: сплавы в современном мире.}}%
			\vspace{5pt}%
		}%
	}
% УСЛОВИЕ В ТАБЛИЦЕ - ИСПРАВЛЕННЫЙ СИНТАКСИС
\noindent    
\colorbox{orange}{\strut\centering\color{white} Условие}%
{\leaders\color{orange}\hrule\hfill}% Линия-заполнитель
\kern0pt
\vspace{5pt}
	
	В современном мире технологии и инженерии сплавы металлов играют важнейшую роль. Эти материалы сочетают свойства различных металлов, создавая уникальные характеристики, которые невозможно достичь с помощью чистых металлов. Сплавы металлов -- это не просто комбинации элементов, а инновационные решения для множества проблем, стоящих перед инженерами и учеными. Использование сплавов дает возможность создавать материалы с уникальным набором физико-химических характеристик, чтобы оптимизировать производственные процессы и повысить эффективность изделий. Это причина того, почему они так широко распространены в самых разнообразных областях: от строительства до высоких технологий.
	
	Силумин (сплав на основе алюминия и кремния) представляет собой важный материал, сочетающий в себе прочность, устойчивость и легкость, что обеспечивает его широкое применение в различных областях промышленности и дизайна. Благодаря прочности, низкому весу, коррозионной стойкости силумин используется в автомобилестроении, авиастроении, кораблестроении, в космической промышленности.
	
	% РИСУНОК 1
	\begin{figure}[H]
		\centering
		\includegraphics[width=0.7\textwidth]{images/image1.jpg} % замените на images/image1.jpg
		\caption*{Рис. 1}
	\end{figure}
	
	Для анализа образца силумина, содержащего алюминий и кремний, массой 30 г его растворили в 400 г 15\%-ного раствора едкого натра, при этом выделился газ, объемом 38,4 л (н.у.). Определите массовую долю алюминия в сплаве (в процентах). Число округлите до целых.
	
	\vspace{10pt}
	
	% РЕШЕНИЕ
	\noindent    
	\colorbox{black}{\strut\centering\color{white} Решение}%
{\leaders\color{black}\hrule\hfill}% Линия-заполнитель
\kern0pt
	
	\vspace{10pt}
	
	Составим уравнения реакций:
	{\fontsize{10}{9}\selectfont
		\begin{align*}
			&2\text{Al} + 2\text{NaOH} + 6\text{H}_2\text{O} \rightarrow 2\text{Na}[\text{Al(OH)}_4] + 3\text{H}_2\uparrow \\
			&\text{Si} + 2\text{NaOH} + \text{H}_2\text{O} \rightarrow \text{Na}_2\text{SiO}_3 + 2\text{H}_2\uparrow \\
			&n(\text{H}_2) = \frac{\scriptstyle 38{,}4\,\text{л}}{\scriptstyle 22{,}4\,\text{л/моль}} = 1{,}714\,\text{моль}\\
			&\text{Если } n(\text{Al}) = x\,\text{моль},\; n(\text{Si}) = y\,\text{моль}, \text{ тогда:} \\
			&27x + 28y = 30 \\
			&1{,}5x + 2y = 1{,}714 \\
			&x = 1\,\text{моль},\; y = 0{,}107\,\text{моль}\\ 
			&m(\text{Al}) = n \cdot M = 1 \cdot 27 = 27\,\text{г} \\
			&\omega(\text{Al}) = \frac{27}{30} = 0{,}9 \text{ или } 90\%
		\end{align*}
	}
	
	\vspace{10pt}
	
	% ОТВЕТ
	\noindent    
	\colorbox{headerblue}{\strut\centering\color{white} Ответ}%
	{\leaders\color{headerblue}\hrule\hfill}% Линия-заполнитель
	\kern0pt
	
	90.
	\vspace{10pt}
	
		
	% ========== ГЕОГРАФИЯ ==========
	\noindent{\fontsize{14}{16.8}\selectfont\bfseries География}\par\vspace{10pt}
	
	% ЗАГОЛОВОК С СИНИМ ФОНОМ И БЕЛЫМ ТЕКСТОМ
	\noindent
	\colorbox{headerblue}{%
		\parbox{\dimexpr\textwidth-2\fboxsep}{%
			\vspace{4pt}
			\centering\color{white}\large\textbf{Задача 1. Дельты рек \hfill 30 баллов}
			\vspace{4pt}
		}%
	}
	
	% ТЕМА С ГОЛУБЫМ ФОНОМ - УБРАТЬ ЛИШНЮЮ СКОБКУ
	\noindent
	\colorbox{lightblue}{%
		\parbox{\dimexpr\textwidth-2\fboxsep}{% Убрана лишняя 1 после width-
			\vspace{5pt}
			{\color{black}\small\textit Тема: региональные физико‑географические особенности; география гидроэнергетики.}}
			\vspace{6pt}
		}%
	}
	
	% УСЛОВИЕ
	\noindent    
	\colorbox{orange}{\strut\centering\color{white} Условие}%
	{\leaders\color{orange}\hrule\hfill} % Линия-заполнитель
	\kern0pt
	
	\vspace{10pt}
	
	На космических снимках представлены дельты крупных рек России. Соотнесите фото и названия рек.
	
	% ТАБЛИЦА - УПРОЩЕННЫЙ ВАРИАНТ
	\vspace{10pt}
	\begin{longtable}{|c|p{0.65\textwidth}|c|}
			\hline
			& \centering\textbf{Снимок} & \textbf{Река} \\
			\hline
			1 &
			\centering
			\includegraphics[width=0.58\linewidth]{images/image2.png} \\
			\vspace{2pt}
			\textbf{Рис. 2}
			\vspace{2pt}
			& Лена \\
			\hline
			2 &
			\centering
			\includegraphics[width=0.58\linewidth]{images/image3.png} \\
			\vspace{2pt}
			\textbf{Рис. 3}
			\vspace{2pt}
			& Волга \\
			\hline
			3 &
			\centering
			\includegraphics[width=0.58\linewidth]{images/image4.png} \\
			\vspace{2pt}
			\textbf{Рис. 4}
			\vspace{2pt}
			& Хатанга \\
			\hline
			4 &
			\centering
			\includegraphics[width=0.5\linewidth]{images/image5.png} \\
			\vspace{2pt}
			\textbf{Рис. 5}
			\vspace{2pt}
			& Обь \\
			\hline
	\end{longtable}
	
	\vspace{10pt}
	
	% ОТВЕТ
	\noindent    
	\colorbox{headerblue}{\strut\centering\color{white} Ответ}%
	{\leaders\color{headerblue}\hrule\hfill} % Линия-заполнитель
	\kern0pt
	
	\vspace{5pt}
	
	Снимок 1 --- Лена; снимок 2 --- Обь; снимок 3 --- Хатанга; снимок 4 --- Волга.
	
\end{document}

